\documentclass{dcbl/challenge}

\usepackage{graphicx}
\usepackage{caption}

\setdoctitle{Basic Navigation in Linux CLI}
\setdocauthor{Stephan Bökelmann}
\setdocemail{sboekelmann@ep1.rub.de}
\setdocinstitute{AG Physik der Hadronen und Kerne}


\begin{document}

The most fundamental way to communicate with our computer is by setting switches on and off. 
While we used to program machines like that just a couple of decades ago, today we can do it with the command line.
The command line interface of a computer is a powerful way to communicate with our machine. 
A teletype keyboard is connected to in input-buffer on our machine. 
We can type in commands, that get stored in the input-buffer. 
By using the return key, a buffered command gets handed over to a process called a shell.
This shell is in charge of dealing with the operating system and makes sure that our commands are executed. 
The shell also handles the output of a given command, storing it's result and displaying additional information in the terminal-emulator.
Being proficient with the terminal is a basic skill for any professional computer user.
In order to be able to do these exercises, you need to have a linux-terminal installed. 
It doesn't matter, whether it is a virtual machine or a native installation of a linux distrubution.

\section*{Exercises}
\begin{aufgabe}
    \begin{figure}
        \centering
        \includegraphics[width=0.5\textwidth]{prompt.png}
        \caption{The prompt of a terminal.}
        \label{fig:prompt}
    \end{figure}
    Open up a terminal. 
    You will be confronted with a line-by-line interface. 
    A line typically consists of the shell signaling to you, that it is ready to exectue a command by writing a couple of characters at the beginning of a line. You can say \textit{it prompts you}, thus we call this start of a line the prompt.
    Even though the specific design and content of the prompt can be configured individually, a typical prompt is shown in figure \ref{fig:prompt}.
    Objects commonly shown in the prompt include:
    \begin{enumerate}
        \item The username of the current user. maxclerkwell in the example.
        \item The hostname of the current machine. SURFnTERF in the example.
        \item The current working directory. $\sim$ in the example.
        \item An arrow indicating that the shell is ready to execute a command. > in the example.
    \end{enumerate}
    Type in any combination of characters into the promt, but do not press the return key.
    Instead press Ctrl+C to abort the current input. 
    Ctrl+C sends the so called \textit{interrupt signal} to the shell.
    You will be using SIGINT a lot, to abort processes that are still running.
    Pressing Ctrl+C on an empty line doesn't have any further side effects. 
    Try it a couple of times and see, what happens.
\end{aufgabe}
\begin{aufgabe}
    When our terminal is filled with a whole bunch of lines we don't want to see no further, a basic command to execute is \textit{clear}. 
    This will clear the screen.
\end{aufgabe}
\begin{aufgabe}
    Most operations in the terminal are text-based operations. 
    Thus or focus for this worksheet is on text-based operations.
    We want to find out, how to read stored information and write it into storage. 
    We will have to distinguish between four types of storages here:
    \begin{enumerate}
        \item The terminal buffer.
        \item The file system.
        \item The shell variables.
        \item The RAM.
    \end{enumerate}
    As soon as something is written to the terminal by the shell, it gets stored in the terminal buffer and subsequently written to the screen. 
    Characters and information shown here are usually not longer modifiable, even though we will get to counter-examples later.
    \begin{itemize}
        \item Use the command \textit{echo "Hello World"} to write something to the terminal-buffer.
    \end{itemize}
    
\end{aufgabe}
\begin{aufgabe}
    The next source of information is the file-system. 
    A file is a named collection of bytes, which can be read from, written to, or both. 
    File names get hierarchically organized. 
    Even though we may think of files being stored in directories, they are not.
    Dennis Ritchie and Ken Thompson wanted to represent the connection between multiple files, as if they would be part of the same thing in the files name. 
    This was already used in the MULTICS OS project.
    Thus they came up with an idea to symbolize this in a file-name-hierarchy, this symbol is a forward-slash (\textit{/}).
    In order to save time writing long filenames, we use the idea of navigating through directories.
    \begin{itemize}
        \item Use the command \textit{cd /} to move to the root-directory of your file-system.
        \item Use the command \textit{ls} to display all content of the current directory.
        \item Use the command \textit{cd /etc} to navigate into the /etc directory.
        \item Use the command \textit{cat passwd} to display the content of the /etc/passwd file.
        \item Use the command \textit{cd $\sim$} to move to your home directory.
        \item Use the command \textit{pwd} to display the current working directory.
    \end{itemize}

\end{aufgabe}
\begin{aufgabe}
    While we sometimes want to store information permanently in the file-system, we sometime also want to store information temporarily.
    For this purpose we use the shell variables.
    Let's assume, that we don't want to forget our favorite number.
    We want to store this number in the shell variable \textit{favorite}.
    \begin{itemize}
        \item Use the command \textit{export favorite=42} to store the value.
        \item Use the command \textit{echo \$favorite} to display the value of the variable.
        \item Use the command \textit{unset favorite} to remove the variable.
        \item Use the command \textit{printenv} to display all shell variables.
    \end{itemize}

\end{aufgabe}
\begin{aufgabe}
    The RAM is storage that can be used by programs running on our machine. 
    The RAM is the main storage for our programs.
    \begin{itemize}
        \item Use the command \textit{free} to display information about the RAM.
        \item Use the command \textit{top} to display the RAM usage of individual processes.
        \item Use \textit{Ctrl+C} to abort the top command.
    \end{itemize}

\end{aufgabe}

\section*{Annotations}
\begin{enumerate}
    \item RedHat Tutorial on navigating the file system:\\ \url{https://www.redhat.com/sysadmin/Linux-file-navigation-commands}
\end{enumerate}

\end{document}
